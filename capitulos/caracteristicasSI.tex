\chapter{Características de SI}

\subsection{Aspectos Tecnológicos, Organizacionais e de Pessoas}

Para uma análise completa das características do Sistema de Informação, faz-se necessário o entendimento das três dimensões essenciais. O projeto desenvolvido é uma relevante contribuição às tecnologias de Internet das Coisas (IoT), e promove uma alternativa de baixo custo, adaptada à realidade e ao mercado nacional.

Do ponto de vista tecnológico, o sistema busca a automatização residencial, onde o usuário pode ter acesso à sua casa remotamente, por meio da operação de um cliente web, com mobilidade, segurança e facilidade de uso. Foram desenvolvidos módulos físicos, cuja implementação tem seu núcleo no componente ESP8266, módulo com capacidade de processamento e radiotransmissão adequada aos padrões IEEE 802.11 -- WiFi. Por meio de transdutores e atuadores, acoplados aos módulos, o sistema obtém dados sobre o estado atual da casa (temperatura, umidade, luminosidade, etc) e pode agir sobre outros componentes (como o acendimento de uma lâmpada, ou a abertura dos portões). A comunicação entre os módulos e os serviços de nuvens passa pelo servidor local, que se conectará diretamente aos servidores remotos por meio de um canal seguro e protegido.

O uso de tecnologias modernas nas plataformas criadas é natural ao usuário, de modo que o seu emprego é transparente, e garante familiaridade e facilidade quando operada. Assim, o cliente não necessita da compra de algum outro dispositivo específico para interagir com sua casa. O seu smartphone, que já é utilizado no dia a dia, entra como o principal mecanismo de controle e troca de informações com a casa inteligente. A criação de uma dashboard responsiva, permite que o painel de controle da casa possa ser acessado tanto por computadores pessoais (PCs) quanto por tablets e smartphones, sem que a experiência do usuário (UX - User Experience) seja afetada pela troca.

Do ponto de vista organizacional, a empresa criada (Hedwig), estará continuamente integrada aos clientes. O conhecimento obtido a partir da análise de dados das casas é utilizado para a melhoria dos sistemas existentes, de modo que possam ser adaptados às necessidades e padrões de uso do cliente, e pode-se desenvolver novos produtos, para suprir demandas específicas, que passarão a ser conhecidas. Um dos maiores desafios organizacionais para a empresa é a segurança dos dados obtidos, bem como da interação entre a casa e o cliente. Um possível ataque, que sequestraria informações confidenciais, ou roubaria o acesso do cliente à casa, pode trazer danos irreparáveis, tanto aos nossos usuários quanto à Hedwig, que perderá a confiança e o prestígio.
Na dimensão de pessoas, o projeto realiza uma quebra de paradigma com a realidade atual dos nossos usuários. Sua casa não é pensada como um local inteligente, integrado com sua rotina, e que pode entender os seus padrões, gostos e preferências. A ruptura promovida, justamente faz frente à histórica visão da casa como simplesmente o seu refúgio diário, de proteção e descanso. Quando fora, não é possível controle ou observação de estado, do que acontece dentro, e quando dentro, tudo que se deseja é feito por meio da interação física com o que se deseja.

Assim, há uma mudança nos costumes de cada um, o que inicialmente trás resistência, natural, mas que deve ser trabalhada para que o sistema se torne tão espontâneo quanto o antigo conceito. A resistência encontrada é, primordialmente, no que diz respeito à segurança. Com todas as cyber ameaças, apresentadas diariamente nos meios de comunicação, o medo de que a sua casa seja tomada por pessoas mal intencionadas se faz presente. Além disso, também há preocupações nos casos de falha, mesmo as que fogem do controle do projeto, como a interrupção no fornecimento de energia elétrica.
Tais desafios devem ser observados e atendidos pelo sistema, de modo que não haja conflito com o usuário, mas trabalho em conjunto para que o todo seja continuamente melhorado, com base nas percepções do cliente e na análise dos dados obtidos.


\subsection{Produção, RH, Finanças/Contabilidade e Vendas/Marketing}

Desde o início, o baixo custo de produção na elaboração do projeto foi uma das prioridades. O baixo custo é determinante porque, com base na realidade nacional, as perspectivas de expansão só podem ser concretizadas, se houver viabilidade para a implementação no maior número possível de residências. Assim, levando-se em conta que a automação de sua casa não é uma prioridade para a maioria das famílias no Brasil, cuja renda é significativamente inferior à renda média de países desenvolvidos, um produto caro, desta natureza, mesmo que ofereça melhores acabamentos, não terá condições de ser adquirido por um número alto de consumidores, fora dos grandes centros.

Os módulos projetados são pequenos, e simples de serem desenvolvidos em escala. A parte mais essencial é o componente ESP8266, conforme explicado anteriormente. Assim, é necessário o contato diretamente com empresas fornecedoras, com o estabelecimento de contratos cujos valores unitários sejam baixos para quantidades elevadas.

Os sistemas de software serão desenvolvidos por uma equipe altamente qualificada e, posteriormente aos testes, distribuídas aos usuários em formas de atualizações. Tais atualizações podem ser gratuitas, para correções ou mudanças que afetam a segurança ou performance, ou pagas, quando há a inserção de novas funcionalidades.

Planeja-se a utilização de softwares de gestão da cadeia de suprimento para o controle e otimização da produção, por meio de um modelo Pull.
O departamento de recursos humanos da empresa (RH) será responsável por atrair novos talentos, alinhados à cultura e às propostas da empresa, bem como na manutenção dos funcionários presente. A principal diferença em relação ao RH é a especialização na busca de profissionais de tecnologia muito bem qualificados. Serão utilizadas ferramentas de Gestão de Relacionamento com o Funcionário (ERM), para que haja maior contato e feedback entre os colaboradores.

O departamento de finanças e contabilidade serão responsáveis por toda a organização fiscal da empresa, bem como pela geração dos relatórios de vendas, custos, compras de matéria prima, etc. Assim como o departamento de Recursos Humanos, os departamentos de finanças e contabilidade farão uso de Aplicações Integradas, fornecidas por empresas responsáveis e de tradição na área.

Todo o processo de venda será analisado internamente, para que possam ser obtidos dados do negócio, bem como um entendimento amplo sobre quem são nossos clientes, quais as suas expectativas, e as motivações para a escolha do produto. Principalmente em termos de marketing, serão buscados acordos com lojas que vendem materiais para construção, frequentadas por pessoas que estão reformando suas residências, e sempre buscam por alguma inovação ou melhoria para sua casa.

Como se trata de um produto novo, ainda não conhecido por grande parte da população, é necessário que também sejam realizadas propagandas, para que haja uma popularização dos conceitos trazidos, bem como para a demonstração das vantagens que a plataforma oferece.

\subsection{Cadeia de Suprimento, Forças Competitivas e Cadeia de Valor}

\subsection{Aspectos Éticos e de Segurança}

\subsection{Comércio Eletrônico}

\subsection{Social Business}

\subsection{Inteligência de Negócios e Gestão de Riscos}